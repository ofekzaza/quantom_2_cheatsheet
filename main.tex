\documentclass[10pt,a4paper]{article}
\usepackage[utf8]{inputenc}
\usepackage[T1]{fontenc}
\usepackage{amsmath,amssymb,amsfonts,amsthm}
\usepackage{multicol}
\usepackage[margin=0.5cm, a4paper]{geometry}
\usepackage{graphicx}
\usepackage{wrapfig}
\usepackage{enumitem}
\usepackage{physics}
\usepackage{xcolor}
\usepackage{titlesec}
\usepackage{empheq}

\newcommand{\important}[1]{\begin{empheq}[box=\fbox]{align*}#1\end{empheq}}


% Compact spacing
\setlength{\parindent}{0pt}
\setlength{\parskip}{0pt}
\setlength{\columnsep}{0.5cm}

% Reduce section spacing
\titlespacing*{\section}{0pt}{2pt}{2pt}
\titlespacing*{\subsection}{0pt}{2pt}{1pt}

% Small text for cheat sheet
\footnotesize

\begin{document}

\begin{multicols*}{3}

\section*{Quantum Mechanics II Cheat Sheet}

% Page 1 Content

\subsection*{Ritz Theorem}
\subsection*{Ritz Theorem}
\important{R[\psi] = \frac{\langle \psi | H | \psi \rangle}{\langle \psi | \psi \rangle} \ge E_0}
\textbf{Step 1: Expand $|\psi\rangle$ in energy eigenbasis}.
$|\psi\rangle = \sum_n c_n |n\rangle, \quad c_n = \langle n | \psi \rangle$.
Then $\langle \psi | \psi \rangle = \sum_n |c_n|^2$.
Using $H|n\rangle = E_n |n\rangle$, $\langle \psi | H | \psi \rangle = \sum_{n,m} c_m^* c_n \langle m | H | n \rangle = \sum_n |c_n|^2 E_n$.
So $R[\psi] = \frac{\sum_n |c_n|^2 E_n}{\sum_n |c_n|^2}$.
\textbf{Step 2: Prove $R[\psi] \ge E_0$}.
Since $E_n \ge E_0$, $\sum_n |c_n|^2 E_n \ge E_0 \sum_n |c_n|^2$.

\subsection*{Thm 3.3.2 Schur's Lemma for Hermitian Ops}
Let $T$ be an IRREP of a finite group $G$ on $V$. If Hermitian operator $C: V \to V$ implies $\forall g \in G, T(g)C = C T(g)$, then $C \propto I$, i.e.,
\important{C = \lambda I, \quad \lambda \in \mathbb{C}}
\textit{Proof:} $C$ Hermitian $\implies \exists$ eigenstate $v \in V$ s.t. $Cv = \lambda v$.
Subspace $span\{T(g)v\}$ is invariant. Since $T$ is IRREP, must be $V$.
$\forall w \in V$, $Cw = C \sum a_g T(g)v = \sum a_g T(g) C v = \lambda w$. Thus $C = \lambda I$.

\subsection*{Thm 4.2.1 Spectrum of Hamiltonian}
\important{H = -\frac{\hbar^2}{2m} \nabla^2 - \frac{a}{r^s}}
with $a>0, s>2$ is unbounded from below.
\textit{Proof:} Scaled state $\Psi(\mathbf{r}) = N e^{-r^2/r_0^2}$.
Expectation values: $\langle T \rangle \sim \frac{\hbar^2}{2mr_0^2}, \langle V \rangle \sim -\frac{a}{r_0^s}$.
$E(r_0) \approx \frac{\alpha}{r_0^2} - \frac{\beta}{r_0^s}$. For $s>2$, as $r_0 \to 0$, $E \to -\infty$.

\subsection*{Thm 3.1.1 Operators equal iff expectations equal}
\important{\langle a | A | a \rangle = \langle a | B | a \rangle \forall |a\rangle \iff A=B}
\textit{Proof:} Use $|\psi\rangle = |a\rangle + |b\rangle$ and $|\psi\rangle = |a\rangle + i|b\rangle$ to show $\langle a|A|b\rangle = \langle a|B|b\rangle$.

\subsection*{Thm 4.1.1 Ritz Theorem}
$E = \frac{\langle \Psi | H | \Psi \rangle}{\langle \Psi | \Psi \rangle} \ge E_1$. Equality iff $\Psi$ is ground state.

\subsection*{Thm 4.1.2 Generalized Ritz}
Expectation value of $H$ is stationary in neighborhood of eigenvalues.
\important{\delta E(\Psi) = 0 \iff H\Psi = E\Psi}

\subsection*{Thm 4.1.3 Variance Theorem}
\important{\sigma^2 = \frac{\langle \Psi | (H-E)^2 | \Psi \rangle}{\langle \Psi | \Psi \rangle} = \langle H^2 \rangle - E^2}
There is at least one eigenval in $[E-\sigma, E+\sigma]$.

\subsection*{Thm 5.4.1 Upper bound on $j_0$}
Given Hilbert space with basis $\{|ab\rangle\}$ of $J^2, J_0$ with eigenvalues $a,b$.
\important{\text{If } a \ge b^2 \implies a - b^2 \ge 0}
\textit{Proof:} $J^2 = J_0^2 + \frac{1}{2}(J_- J_+ + J_+ J_-) \implies a - b^2 \ge 0$.

\subsection*{Thm 3.1.2 Ops equal within phase}
\important{A = e^{i\theta} B \iff |\langle a|A|b\rangle| = |\langle a|B|b\rangle|}
\textit{Proof "$\Leftarrow$":} $A|b_j\rangle = e^{i\theta_j} B|b_j\rangle$. Apply to $|b_1\rangle + |b_2\rangle$.
$A(|b_1\rangle+|b_2\rangle) = e^{i\theta_{12}} B(|b_1\rangle+|b_2\rangle) = e^{i\theta_1} B|b_1\rangle + e^{i\theta_2} B|b_2\rangle$.
Linearity $\implies B(e^{i\theta_{12}} - e^{i\theta_1})|b_1\rangle + B(e^{i\theta_{12}} - e^{i\theta_2})|b_2\rangle = 0$.
Inner product with $B|b_i\rangle \implies e^{i\theta_1} = e^{i\theta_2}$. Phase is global.

\subsection*{Thm 3.1.3 Scalar Product Preserving}
If $T: \mathcal{V} \to \mathcal{V}$ preserves scalar product magnitude
\important{|\langle \phi | \psi \rangle| = |\langle T\phi | T\psi \rangle|}
then $T$ is unitary or anti-unitary. (Wigner's Theorem).

\subsection*{Thm 4.2.2 For $s<2$ spectrum of H}
$H = -\frac{\hbar^2}{2m}\nabla^2 - \frac{a}{r^s}$ ($a>0$) contains infinite bound states.
\textit{Proof:} Trial $\Psi(r) = N e^{-(r-r_0)^2/\beta^2 r_0^2}$.
For large $r_0$, $E < 0$ is possible.

\subsection*{Thm 8.2.2 Triangular Rule}
Admissible $j$ are
\important{|j_1-j_2| \le j \le j_1+j_2}




% Page 2 Content

\subsection*{Thm 12.3.1 Ops S and A}
Operators $\mathcal{S}$ and $\mathcal{A}$ satisfy:
(a) $\mathcal{S}^\dagger = \mathcal{S}, \mathcal{A}^\dagger = \mathcal{A}$.
(b) Commute with $P_g$ for all $g \in S_N$.
\important{P_g \mathcal{S} = \mathcal{S}, \quad P_g \mathcal{A} = \text{sign}(g) \mathcal{A}}
(c) Orthogonal projectors of $\mathcal{H}_{so}$.
\important{\mathcal{S}^2=\mathcal{S}, \quad \mathcal{A}^2=\mathcal{A}, \quad \mathcal{S}\mathcal{A}=\mathcal{A}\mathcal{S}=0}

\subsection*{Landau Levels Derivation (Handwritten)}
$H = \frac{1}{2m}(\mathbf{p} - q\mathbf{A})^2$. $B = B\hat{z}$. Gauge $\mathbf{A} = (-By, 0, 0)$.
$H = \frac{p_y^2}{2m} + \frac{1}{2}m \omega_c^2 (y - y_0)^2 + \frac{\hbar^2 k_z^2}{2m}$.
Harmonic oscillator centered at $y_0 = -\frac{\hbar k}{qB}$. $\omega_c = \frac{qB}{m}$.
\important{E = \hbar\omega_c (n + \frac{1}{2}) + \frac{\hbar^2 k_z^2}{2m}}

\subsection*{Thm 13.9.1 Optical Theorem}
\important{\sigma_{tot}(k) = \frac{4\pi}{k} \text{Im} f_k(0)}
\textit{Proof:} $f_k(\theta) = f(k,k') = -\frac{m}{2\pi\hbar^2} (2\pi\hbar)^3 \langle \mathbf{k} | T | \mathbf{k} \rangle$.
Use Lippmann-Schwinger: $\text{Im}\langle \mathbf{k}|T|\mathbf{k}\rangle = \text{Im}\langle \mathbf{k}|V|\Psi_\mathbf{k}^{in}\rangle$.
Principal value integral contour (semicircle over pole).
$\frac{1}{E - H_0 + i\epsilon} = \text{Pr}\frac{1}{E-H_0} - i\pi \delta(E-H_0)$.
Result: $\text{Im} f_k(0) = \frac{k}{4\pi} \int d\Omega' |f_k(\theta')|^2 = \frac{k}{4\pi} \sigma_{tot}$.

\subsection*{Thm 8.5.2 Eigenstates of $J^2, J_0$}
Transformation matrix elements only depend on $a, \beta$ (independent of $j,m$).
\important{\langle \Phi_{a j m} | \Psi_{\beta j m} \rangle = \delta_{jj'} \delta_{mm'} \langle \Phi_{a j} || \Psi_{\beta j} \rangle}

\subsection*{CG Coefficients Example (Handwritten)}
Addition of angular momentum $J = J_1 + J_2$.
Max $m = j_1 + j_2$ is unique: $|j_1 j_2; j_1+j_2, j_1+j_2\rangle = |j_1, j_1\rangle |j_2, j_2\rangle$.
Apply $J_- = J_{1-} + J_{2-}$ to find lower $m$ states.
\important{\alpha = \sqrt{\frac{j_1}{j_1+j_2}}, \quad \beta = -\sqrt{\frac{j_2}{j_1+j_2}}}

% Page 3 Content

\subsection*{Hydrogen-like Half-Space}
$V(z) = -Ze^2/z (z>0), \infty (z<0)$.
Solving SE gives Rydberg states for odd parity (wavefunc must vanish at $z=0$).
\important{E_n = -\frac{Z^2 e^4 m}{2\hbar^2 n^2}}

\subsection*{Spin-Orbit Coupling}
$H = \frac{\alpha}{\hbar^2} \mathbf{L} \cdot \mathbf{S} = \frac{\alpha}{2\hbar^2}(J^2 - L^2 - S^2)$.
Energy shift:
\important{\Delta E = \frac{\alpha}{2} (j(j+1) - l(l+1) - s(s+1))}
Splitting between states $j = l \pm 1/2$.

\subsection*{Angular Momentum Operators}
$L_z |l,m\rangle = \hbar m |l,m\rangle, \quad L^2 |l,m\rangle = \hbar^2 l(l+1) |l,m\rangle$.
\important{L_\pm |l,m\rangle = \hbar \sqrt{l(l+1) - m(m\pm 1)} |l, m\pm 1\rangle}
$L_x = \frac{1}{2}(L_+ + L_-), \quad L_y = \frac{1}{2i}(L_+ - L_-)$.

\subsection*{Infinite Spherical Well}
$V(r) = 0 (r<R), \infty (r>R)$.
Sol: $R_{nl}(r) = A j_l(k_n r)$. Boundary $j_l(k_n R) = 0$.
\important{E_{nl} = \frac{\hbar^2 k_{nl}^2}{2m}, \quad j_l(k_{nl} R) = 0}

\subsection*{Permutation Group Representations ($S_3$)}
Matrices for basis vectors (possibly defining specific rep):
$T(123) = \begin{pmatrix} 0 & 0 & 1 \\ 1 & 0 & 0 \\ 0 & 1 & 0 \end{pmatrix}$, $T(12) = \begin{pmatrix} 0 & 1 & 0 \\ 1 & 0 & 0 \\ 0 & 0 & 1 \end{pmatrix}$.
Eigenvalues of $T(123)$: $\det(T - \lambda I) = -\lambda^3 + 1 = 0 \implies \lambda = 1, e^{\pm i 2\pi/3}$.
Vectors: $|1\rangle = \frac{1}{\sqrt{3}}(1,1,1)^T$ (invariant).

\subsection*{Spin in Magnetic Field (Rabi)}
$H = -\gamma \mathbf{S} \cdot \mathbf{B}$. Let $\mathbf{B} = B_0 \hat{z} + B_1 (\cos\omega t \hat{x} + \sin\omega t \hat{y})$.
Transition probability (Rabi formula):
\important{P(t) = \frac{\Omega^2}{\Omega^2 + \Delta^2} \sin^2\left( \frac{\sqrt{\Omega^2+\Delta^2}t}{2} \right)}
Where $\Delta = \omega - \omega_0$ (detuning), $\Omega = \gamma B_1$ (Rabi freq).

% Page 4 Content

\subsection*{Tensor Operators Example}
Consider $V \propto (x^2 - y^2)$.
Selection rules for matrix elements $\langle l', m' | V | l, m \rangle$:
If $V$ transforms like $T_k^q$ (here $k=2, q=\pm 2$), then $m' = m + q$.
\important{\Delta m = \pm 2}
Matrix form example:
$V \doteq \begin{pmatrix} 0 & 0 & \alpha \\ 0 & 0 & 0 \\ \alpha^* & 0 & 0 \end{pmatrix}$.

\subsection*{Time-Dependent Perturbation (HO)}
$V(t) = F_0 x (0 < t < T)$.
First order amp: $c_n^{(1)}(t) = -\frac{i}{\hbar} \int_0^t dt' \langle n | V(t') | i \rangle e^{i\omega_{ni}t'}$.
For HO $x \propto (a + a^\dagger)$. Selection: $\langle n | x | m \rangle \neq 0$ only if $n = m \pm 1$.
Transition $0 \to 1$:
\important{c_1^{(1)}(t) = -\frac{i}{\hbar} F_0 \sqrt{\frac{\hbar}{2m\omega}} \frac{e^{i\omega t}-1}{i\omega}}

\subsection*{Identical Particles}
Two particles in 1D box. 
\important{E = \frac{\hbar^2\pi^2}{2mL^2}(n_1^2 + n_2^2)}
$\Psi(x_1, x_2) = \frac{1}{\sqrt{2}} [\psi_{n_1}(x_1)\psi_{n_2}(x_2) \pm \psi_{n_1}(x_2)\psi_{n_2}(x_1)] \chi_{spin}$.

Spin States:
- Triplet ($S=1$): Symmetric. Requires Antisym spatial (Fermions) or Sym spatial (Bosons).
- Singlet ($S=0$): Antisymmetric. Requires Sym spatial (Fermions) or Antisym spatial (Bosons).

\subsection*{3 Particles in Harmonic Oscillator}
Hamiltonian $H = \sum_{i=1}^3 \frac{p_i^2}{2m} + \frac{1}{2}m\omega x_i^2$.
Energy $E = \hbar\omega(n_1+n_2+n_3 + \frac{3}{2})$.
Example: 3 identical fermions ($s=1/2$, polarized spin). Spatial part must be totally antisymmetric.
Ground State: $n_1=0, n_2=1, n_3=2$.
\important{E_0 = \frac{9}{2}\hbar\omega, \quad \Psi_{GS} = \frac{1}{\sqrt{3!}} \det | \psi_{n_i}(x_j) |}

\subsection*{Slater Determinant}
N-fermion state:
\important{\Psi(1,\dots,N) = \frac{1}{\sqrt{N!}} \det(\psi_i(j))}
Zero if two particles in same state (Pauli exclusion).

\end{multicols*}

\end{document}

% new content

\subsection*{Hydrogen Atom Solution}
\important{\Psi_{nlm} = N \rho^l e^{-\rho/2} L_{n-l-1}^{2l+1}(\rho) Y_{lm}(\theta,\phi), \quad E_n = - \frac{Z^2 \mu e^4}{32\pi^2\epsilon_0^2\hbar^2 n^2}}
\textbf{1. Separation:} $\Psi(\mathbf{r}) = R(r)Y_{lm}(\Omega)$. Radial eq for $u(r) = rR(r)$.
\textbf{2. Radial Eq:} Dimensionless $\rho = \frac{2Z}{na_0}r$, $a_0 = \frac{4\pi\epsilon_0\hbar^2}{\mu e^2}$.
\important{u''(\rho) + \left[ \frac{n}{\rho} - \frac{1}{4} - \frac{l(l+1)}{\rho^2} \right] u(\rho) = 0}
\textbf{3. Asymptotics:} $\rho \to 0: u \sim \rho^{l+1}$. $\rho \to \infty: u \sim e^{-\rho/2}$.
\textbf{4. Ansatz:} $u(\rho) = \rho^{l+1} e^{-\rho/2} w(\rho)$.
Kummer's Eq for $w(\rho)$: $\rho w'' + (2l+2-\rho)w' + (n-l-1)w = 0$.
\textbf{5. Series Solution:} $w(\rho) = \sum a_k \rho^k$. Recurrence relation:
\important{a_{k+1} = \frac{k+l+1-n}{(k+1)(k+2l+2)} a_k}
\textbf{6. Quantization:} Series must terminate for normalizable solution.
Numerator vanishes at $k = p_{max} \implies p_{max} + l + 1 = n$ (integer).
Result: Associated Laguerre Polynomials $L_{n-l-1}^{2l+1}(\rho)$.
