\documentclass[10pt,a4paper]{article}
\usepackage[utf8]{inputenc}
\usepackage[T1]{fontenc}
\usepackage{amsmath,amssymb,amsfonts,amsthm}
\usepackage{multicol}
\usepackage[left=0.5cm, right=0.5cm, top=1.4cm, bottom=0.5cm, a4paper]{geometry}
\usepackage{graphicx}
\usepackage{wrapfig}
\usepackage{enumitem}
\usepackage{physics}
\usepackage{xcolor}
\usepackage{titlesec}
\usepackage{empheq}
\usepackage{fancyhdr}

\pagestyle{fancy}
\fancyhf{} % clear all fields
\fancyhead[C]{\thepage} % Centered page number
\renewcommand{\headrulewidth}{0pt}
\setlength{\headsep}{0cm} % Space between header and text

\newcommand{\important}[1]{\begin{empheq}[box=\fbox]{align*}#1\end{empheq}}


% Compact spacing
\setlength{\parindent}{0pt}
\setlength{\parskip}{0pt}
\setlength{\columnsep}{0.5cm}

% Reduce section spacing
\titlespacing*{\section}{0pt}{2pt}{2pt}
\titlespacing*{\subsection}{0pt}{2pt}{1pt}

% Small text for cheat sheet
\footnotesize

\begin{document}

\begin{multicols*}{3}

\section*{Quantum Mechanics II}

% Page 1 Content

\subsection*{Ritz Theorem}
\important{R[\psi] = \frac{\langle \psi | \mathcal{H} | \psi \rangle}{\langle \psi | \psi \rangle} \ge E_0}
\textbf{Step 1: Expand $|\psi\rangle$ in energy eigenbasis}.
$|\psi\rangle = \sum_n c_n |n\rangle, \quad c_n = \langle n | \psi \rangle$.
Then $\langle \psi | \psi \rangle = \sum_n |c_n|^2$.
Using $\mathcal{H}|n\rangle = E_n |n\rangle$, $\langle \psi | \mathcal{H} | \psi \rangle = \sum_{n,m} c_m^* c_n \langle m | \mathcal{H} | n \rangle = \sum_n |c_n|^2 E_n$.
So $R[\psi] = \frac{\sum_n |c_n|^2 E_n}{\sum_n |c_n|^2}$.
\textbf{Step 2: Prove $R[\psi] \ge E_0$}.
Since $E_n \ge E_0$, $\sum_n |c_n|^2 E_n \ge E_0 \sum_n |c_n|^2$.

\subsection*{Thm 3.3.2 Schur's Lemma for Hermitian Ops}
Let $T$ be an IRREP of a finite group $G$ on $V$. If Hermitian operator $C: V \to V$ implies $\forall g \in G, T(g)C = C T(g)$, then $C \propto I$, i.e.,
\important{C = \lambda I, \quad \lambda \in \mathbb{C}}
\textit{Proof:} $C$ Hermitian $\implies \exists$ eigenstate $v \in V$ s.t. $Cv = \lambda v$.
Subspace $span\{T(g)v\}$ is invariant. Since $T$ is IRREP, must be $V$.
$\forall w \in V$, $Cw = C \sum a_g T(g)v = \sum a_g T(g) C v = \lambda w$. Thus $C = \lambda I$.

\subsection*{Thm 4.2.1 Spectrum of Hamiltonian}
\important{\mathcal{H} = -\frac{\hbar^2}{2m} \nabla^2 - \frac{a}{r^s}}
with $a>0, s>2$ is unbounded from below.
\textit{Proof:} Scaled state $\Psi(\mathbf{r}) = N e^{-r^2/r_0^2}$.
Expectation values: $\langle T \rangle \sim \frac{\hbar^2}{2mr_0^2}, \langle V \rangle \sim -\frac{a}{r_0^s}$.
$E(r_0) \approx \frac{\alpha}{r_0^2} - \frac{\beta}{r_0^s}$. For $s>2$, as $r_0 \to 0$, $E \to -\infty$.

\subsection*{Thm 3.1.1 Operators equal iff expectations equal}
\important{\langle a | A | a \rangle = \langle a | B | a \rangle \forall |a\rangle \iff A=B}
\textit{Proof:} Use $|\psi\rangle = |a\rangle + |b\rangle$ and $|\psi\rangle = |a\rangle + i|b\rangle$ to show $\langle a|A|b\rangle = \langle a|B|b\rangle$.

\subsection*{Thm 4.1.1 Ritz Theorem}
$E = \frac{\langle \Psi | \mathcal{H} | \Psi \rangle}{\langle \Psi | \Psi \rangle} \ge E_1$. Equality iff $\Psi$ is ground state.

\subsection*{Thm 4.1.2 Generalized Ritz}
Expectation value of $H$ is stationary in neighborhood of eigenvalues.
\important{\delta E(\Psi) = 0 \iff \mathcal{H}\Psi = E\Psi}

\subsection*{Thm 4.1.3 Variance Theorem}
\important{\sigma^2 = \frac{\langle \Psi | (\mathcal{H}-E)^2 | \Psi \rangle}{\langle \Psi | \Psi \rangle} = \langle \mathcal{H}^2 \rangle - E^2}
There is at least one eigenval in $[E-\sigma, E+\sigma]$.

\subsection*{Thm 5.4.1 Upper bound on $j_0$}
Given Hilbert space with basis $\{|ab\rangle\}$ of $J^2, J_0$ with eigenvalues $a,b$.
\important{\text{If } a \ge b^2 \implies a - b^2 \ge 0}
\textit{Proof:} $J^2 = J_0^2 + \frac{1}{2}(J_- J_+ + J_+ J_-) \implies a - b^2 \ge 0$.

\subsection*{Thm 3.1.2 Ops equal within phase}
\important{A = e^{i\theta} B \iff |\langle a|A|b\rangle| = |\langle a|B|b\rangle|}
\textit{Proof "$\Leftarrow$":} $A|b_j\rangle = e^{i\theta_j} B|b_j\rangle$. Apply to $|b_1\rangle + |b_2\rangle$.
$A(|b_1\rangle+|b_2\rangle) = e^{i\theta_{12}} B(|b_1\rangle+|b_2\rangle) = e^{i\theta_1} B|b_1\rangle + e^{i\theta_2} B|b_2\rangle$.
Linearity $\implies B(e^{i\theta_{12}} - e^{i\theta_1})|b_1\rangle + B(e^{i\theta_{12}} - e^{i\theta_2})|b_2\rangle = 0$.
Inner product with $B|b_i\rangle \implies e^{i\theta_1} = e^{i\theta_2}$. Phase is global.

\subsection*{Thm 3.1.3 Scalar Product Preserving}
If $T: \mathcal{V} \to \mathcal{V}$ preserves scalar product magnitude
\important{|\langle \phi | \psi \rangle| = |\langle T\phi | T\psi \rangle|}
then $T$ is unitary or anti-unitary. (Wigner's Theorem).

\subsection*{Thm 4.2.2 For $s<2$ spectrum of H}
$\mathcal{H} = -\frac{\hbar^2}{2m}\nabla^2 - \frac{a}{r^s}$ ($a>0$) contains infinite bound states.
\textit{Proof:} Trial $\Psi(r) = N e^{-(r-r_0)^2/\beta^2 r_0^2}$.
For large $r_0$, $E < 0$ is possible.

\subsection*{Thm 8.2.2 Triangular Rule}
Admissible $j$ are
\important{|j_1-j_2| \le j \le j_1+j_2}




% Page 2 Content

\subsection*{Thm 12.3.1 Ops S and A}
Operators $\mathcal{S}$ and $\mathcal{A}$ satisfy:
(a) $\mathcal{S}^\dagger = \mathcal{S}, \mathcal{A}^\dagger = \mathcal{A}$.
(b) Commute with $P_g$ for all $g \in S_N$.
\important{P_g \mathcal{S} = \mathcal{S}, \quad P_g \mathcal{A} = \text{sign}(g) \mathcal{A}}
(c) Orthogonal projectors of $\mathcal{H}_{so}$.
\important{\mathcal{S}^2=\mathcal{S}, \quad \mathcal{A}^2=\mathcal{A}, \quad \mathcal{S}\mathcal{A}=\mathcal{A}\mathcal{S}=0}

\subsection*{Landau Levels Derivation (Handwritten)}
$\mathcal{H} = \frac{1}{2m}(\mathbf{p} - q\mathbf{A})^2$. $B = B\hat{z}$. Gauge $\mathbf{A} = (-By, 0, 0)$.
$\mathcal{H} = \frac{p_y^2}{2m} + \frac{1}{2}m \omega_c^2 (y - y_0)^2 + \frac{\hbar^2 k_z^2}{2m}$.
Harmonic oscillator centered at $y_0 = -\frac{\hbar k}{qB}$. $\omega_c = \frac{qB}{m}$.
\important{E = \hbar\omega_c (n + \frac{1}{2}) + \frac{\hbar^2 k_z^2}{2m}}

\subsection*{Thm 13.9.1 Optical Theorem}
\important{\sigma_{tot}(k) = \frac{4\pi}{k} \text{Im} f_k(0)}
\textit{Proof:} $f_k(\theta) = f(k,k') = -\frac{m}{2\pi\hbar^2} (2\pi\hbar)^3 \langle \mathbf{k} | T | \mathbf{k} \rangle$.
Use Lippmann-Schwinger: $\text{Im}\langle \mathbf{k}|T|\mathbf{k}\rangle = \text{Im}\langle \mathbf{k}|V|\Psi_\mathbf{k}^{in}\rangle$.
Principal value integral contour (semicircle over pole).
$\frac{1}{E - \mathcal{H}_0 + i\epsilon} = \text{Pr}\frac{1}{E-\mathcal{H}_0} - i\pi \delta(E-\mathcal{H}_0)$.
Result: $\text{Im} f_k(0) = \frac{k}{4\pi} \int d\Omega' |f_k(\theta')|^2 = \frac{k}{4\pi} \sigma_{tot}$.





% Page 3 Content

\subsection*{Two Body Problem}
System of 2 particles with potential $V(|\mathbf{r}_1 - \mathbf{r}_2|)$.
Separation into Center of Mass ($\mathbf{R}$) and Relative ($\mathbf{r}$) motion.
$\mathbf{R} = \frac{m_1\mathbf{r}_1+m_2\mathbf{r}_2}{m_1+m_2}, \quad \mathbf{r} = \mathbf{r}_1-\mathbf{r}_2$.
Total Mass $M = m_1+m_2$, Reduced Mass $\mu = \frac{m_1m_2}{m_1+m_2}$.
Hamiltonian separates: $\mathcal{H} = \mathcal{H}_{CM} + \mathcal{H}_{rel}$.
\important{\mathcal{H}_{CM} = \frac{\mathbf{P}^2}{2M}, \quad \mathcal{H}_{rel} = \frac{\mathbf{p}^2}{2\mu} + V(r)}
$\Psi(\mathbf{R}, \mathbf{r}) = e^{i\mathbf{K}\cdot\mathbf{R}} \psi(\mathbf{r})$. $E_{tot} = E_{CM} + E_{rel}$.
$\mathcal{H}_{rel}$ is effectively a 1-body problem with mass $\mu$.

\subsection*{Hydrogen Atom Solution}
\important{\Psi_{nlm} = N \rho^l e^{-\rho/2} L_{n-l-1}^{2l+1}(\rho) Y_{lm}(\theta,\phi)}
\important{E_n = - \frac{Z^2 \mu e^4}{32\pi^2\epsilon_0^2\hbar^2 n^2}}
\textbf{1. Separation:} $\Psi(\mathbf{r}) = R(r)Y_{lm}(\Omega)$. Radial eq for $u(r) = rR(r)$.
\textbf{2. Radial Eq:} Dimensionless $\rho = \frac{2Z}{na_0}r$, $a_0 = \frac{4\pi\epsilon_0\hbar^2}{\mu e^2}$.
\important{u''(\rho) + \left[ \frac{n}{\rho} - \frac{1}{4} - \frac{l(l+1)}{\rho^2} \right] u(\rho) = 0}
\textbf{3. Asymptotics:} $\rho \to 0: u \sim \rho^{l+1}$. $\rho \to \infty: u \sim e^{-\rho/2}$.
\textbf{4. Ansatz:} $u(\rho) = \rho^{l+1} e^{-\rho/2} w(\rho)$.
Kummer's Eq for $w(\rho)$: $\rho w'' + (2l+2-\rho)w' + (n-l-1)w = 0$.
\textbf{5. Series Solution:} $w(\rho) = \sum a_k \rho^k$. Recurrence relation:
\important{a_{k+1} = \frac{k+l+1-n}{(k+1)(k+2l+2)} a_k}
\textbf{6. Quantization:} Series must terminate for normalizable solution.
Numerator vanishes at $k = p_{max} \implies p_{max} + l + 1 = n$ (integer).

\textbf{Definitions:} 

$Z$ Atomic number (Nucleus charge $+Ze$).

$L_q^p(x)$ Assoc. Laguerre Polys: 

$L_q^p(x) = \frac{x^{-p} e^x}{q!} \frac{d^q}{dx^q}(e^{-x} x^{p+q})$.

\subsection*{Hydrogen-like Half-Space}
$V(z) = -Ze^2/z (z>0), \infty (z<0)$.

Solving SE gives Rydberg states for odd parity (wavefunc must vanish at $z=0$).
\important{E_n = -\frac{Z^2 e^4 m}{2\hbar^2 n^2}}

\subsection*{Spin-Orbit Coupling}
$\mathcal{H} = \frac{\alpha}{\hbar^2} \mathbf{L} \cdot \mathbf{S} = \frac{\alpha}{2\hbar^2}(J^2 - L^2 - S^2)$.
Energy shift:
\important{\Delta E = \frac{\alpha}{2} (j(j+1) - l(l+1) - s(s+1))}
Splitting between states $j = l \pm 1/2$.



\subsection*{Infinite Spherical Well}
$V(r) = 0 (r<R), \infty (r>R)$.
Sol: $R_{nl}(r) = A j_l(k_n r)$. Boundary $j_l(k_n R) = 0$.
\important{E_{nl} = \frac{\hbar^2 k_{nl}^2}{2m}, \quad j_l(k_{nl} R) = 0}

\subsection*{Permutation Group Representations ($S_3$)}
Matrices for basis vectors (possibly defining specific rep):
$T(123) = \begin{pmatrix} 0 & 0 & 1 \\ 1 & 0 & 0 \\ 0 & 1 & 0 \end{pmatrix}$, $T(12) = \begin{pmatrix} 0 & 1 & 0 \\ 1 & 0 & 0 \\ 0 & 0 & 1 \end{pmatrix}$.
Eigenvalues of $T(123)$: $\det(T - \lambda I) = -\lambda^3 + 1 = 0 \implies \lambda = 1, e^{\pm i 2\pi/3}$.
Vectors: $|1\rangle = \frac{1}{\sqrt{3}}(1,1,1)^T$ (invariant).

\subsection*{Spin in Magnetic Field (Rabi)}
$\mathcal{H} = -\gamma \mathbf{S} \cdot \mathbf{B}$. Let $\mathbf{B} = B_0 \hat{z} + B_1 (\cos\omega t \hat{x} + \sin\omega t \hat{y})$.
Transition probability (Rabi formula):
\important{P(t) = \frac{\Omega^2}{\Omega^2 + \Delta^2} \sin^2\left( \frac{\sqrt{\Omega^2+\Delta^2}t}{2} \right)}
Where $\Delta = \omega - \omega_0$ (detuning), $\Omega = \gamma B_1$ (Rabi freq).

% Page 4 Content



\subsection*{Time-Dependent Perturbation (HO)}
$V(t) = F_0 x (0 < t < T)$.
First order amp: $c_n^{(1)}(t) = -\frac{i}{\hbar} \int_0^t dt' \langle n | V(t') | i \rangle e^{i\omega_{ni}t'}$.
For HO $x \propto (a + a^\dagger)$. Selection: $\langle n | x | m \rangle \neq 0$ only if $n = m \pm 1$.
Transition $0 \to 1$:
\important{c_1^{(1)}(t) = -\frac{i}{\hbar} F_0 \sqrt{\frac{\hbar}{2m\omega}} \frac{e^{i\omega t}-1}{i\omega}}

\subsection*{Identical Particles}
Two particles in 1D box. 
\important{E = \frac{\hbar^2\pi^2}{2mL^2}(n_1^2 + n_2^2)}
$\Psi(x_1, x_2) = \frac{1}{\sqrt{2}} [\psi_{n_1}(x_1)\psi_{n_2}(x_2) \pm \psi_{n_1}(x_2)\psi_{n_2}(x_1)] \chi_{spin}$.

Spin States:
- Triplet ($S=1$): Symmetric. Requires Antisym spatial (Fermions) or Sym spatial (Bosons).
- Singlet ($S=0$): Antisymmetric. Requires Sym spatial (Fermions) or Antisym spatial (Bosons).

\subsection*{3 Particles in Harmonic Oscillator}
Hamiltonian $\mathcal{H} = \sum_{i=1}^3 \frac{p_i^2}{2m} + \frac{1}{2}m\omega x_i^2$.
Energy $E = \hbar\omega(n_1+n_2+n_3 + \frac{3}{2})$.
Example: 3 identical fermions ($s=1/2$, polarized spin). Spatial part must be totally antisymmetric.
Ground State: $n_1=0, n_2=1, n_3=2$.
\important{E_0 = \frac{9}{2}\hbar\omega, \quad \Psi_{GS} = \frac{1}{\sqrt{3!}} \det | \psi_{n_i}(x_j) |}

\subsection*{Slater Determinant}
N-fermion state:
\important{\Psi(1,\dots,N) = \frac{1}{\sqrt{N!}} \det(\psi_i(j))}
Zero if two particles in same state (Pauli exclusion).

\section*{Tirgul Summaries}

\subsection*{Symmetry \& Conservation}
\textbf{Noether's Theorem:} Continuous symmetry $\implies$ conservation.
- Time Translation ($t \to t+\epsilon$) $\implies$ Energy ($\mathcal{H}$) cons.
- Space Translation ($\mathbf{r} \to \mathbf{r}+\mathbf{a}$) $\implies$ Momentum ($\mathbf{p}$) cons.
- Rotation ($\hat{n}, \theta$) $\implies$ Angular Momentum ($\mathbf{L}$) cons.
If $S$ invariant under $q_i \to q_i + \epsilon \mathcal{F}_i$, conserved charge $Q = \sum \frac{\partial \mathcal{L}}{\partial \dot{q}_i} \mathcal{F}_i$.


\subsection*{Discrete Symmetries}

usage: $T^\dagger A T=A_{reversed}$

\textbf{Parity ($\pi, P, \prod$):} Unitary, $\pi^\dagger = \pi^{-1} = \pi$.
Action: $\mathbf{r} \to -\mathbf{r}, \quad \mathbf{p} \to -\mathbf{p}$.
$\mathbf{L} = \mathbf{r} \times \mathbf{p} \to (-\mathbf{r}) \times (-\mathbf{p}) = \mathbf{L}$ (Pseudovector).
Spin $\mathbf{S} \to \mathbf{S}$.
Eigenvalues $\pm 1$. Limits: $\pi |l,m\rangle = (-1)^l |l,m\rangle$.

\textbf{Time Reversal ($\Theta$ or $T$):} Anti-unitary ($T i T^{-1} = -i$).
Action: $t \to -t, \quad \mathbf{r} \to \mathbf{r}, \quad \mathbf{p} \to -\mathbf{p}$.
$\mathbf{L} \to -\mathbf{L}, \quad \mathbf{S} \to -\mathbf{S}$.
Kramers Degeneracy: For half-integer $j$, $T^2 = -1 \implies$ degeneracy.

\textbf{Rotation Group $SO(3)$:} 

$R^T R = I, \det R = 1$.
Preserves inner product $\mathbf{v} \cdot \mathbf{u}$.
Generator of rotation is Angular Momentum $\mathbf{L}$.
$R(\hat{n}, \theta) = e^{-i \theta \mathbf{n} \cdot \mathbf{L} / \hbar}$.
\textbf{Rotation Matrices (Passive $x \to R \hat{x}$):}
$R_z(\theta) = \begin{pmatrix} \cos\theta & \sin\theta & 0 \\ -\sin\theta & \cos\theta & 0 \\ 0 & 0 & 1 \end{pmatrix}$

$ R_y(\theta) = \begin{pmatrix} \cos\theta & 0 & -\sin\theta \\ 0 & 1 & 0 \\ \sin\theta & 0 & \cos\theta \end{pmatrix}$ note

$R_x(\theta) = \begin{pmatrix} 1 & 0 & 0 \\ 0 & \cos\theta & \sin\theta \\ 0 & -\sin\theta & \cos\theta \end{pmatrix}$ signs

$RR^T=I \quad \det R=1\\ R=\sum_{a>b}^di\omega_{ab}X_{ab} \quad \omega = \sum_{a < b}^n i \omega_{ab} X_{ab} \\ (X_{ab})_{ij} = i(\delta_{ai}\delta_{bj} - \delta_{aj}\delta_{bi})$

$[X_{ab}, X_{cd}] = i\hbar( \delta_{bc} X_{ad} - \delta_{bd} X_{ac} - \delta_{ac} X_{bd} + \delta_{ad} X_{bc})$

Generators $G_i$: $R(\hat{n}, d\theta) \approx 1 - \frac{i}{\hbar} d\theta \hat{n}\cdot\vec{J}$.
$J_k = -i\hbar \epsilon_{ijk} x_i \partial_j$. (orbital)

Spin Generators $\vec{S}$: $S_i = \frac{\hbar}{2}\sigma_i$

Rotation of Spinor: $D^{(1/2)}(\hat{n}, \theta) = e^{-i\theta \hat{n}\cdot\vec{S}/\hbar} = \cos\frac{\theta}{2}I - i\vec{n}\cdot\vec{\sigma}\sin\frac{\theta}{2}$.

\subsection*{Angular Momentum}
\textbf{Definitions:} $\vec{L} = \vec{r} \times \vec{p}$, $\vec{J} = \vec{L} + \vec{S} \\ L_i=\epsilon_{ijk}X_jP_k$

Rotation Generator: $R(\hat{n}, \theta) = e^{-i\theta \hat{n}\cdot\vec{J}/\hbar}$

$L_i=\frac{1}{2}\varepsilon_{ilk}X_{lk}$ (i goes from 1 to 3)

\textbf{Commutators:}

$[J_i, J_j] = i\hbar \epsilon_{ijk} J_k, \quad [L^2, L_i] = 0$.
$[L_i, x_j] = i\hbar \epsilon_{ijk} x_k, \quad [L_i, p_j] = i\hbar \epsilon_{ijk} p_k$.

$J_1=X_{32},J_2=X_{13},J_3=X_{21}$ \underbar{(3d):}

$\quad [J_1, J_2] = i\hbar J_3, \quad [J_2, J_3] = i\hbar J_1$

$\quad [J_3, J_1] = i\hbar J_2 \quad [J_a, J_b] = i\hbar \epsilon_{abc}J_c$

\textbf{Eigenvalues \& Ladder:}

$L^2 |l,m\rangle = \hbar^2 l(l+1) |l,m\rangle, \\ 
L_z |l,m\rangle = \hbar m |l,m\rangle \\
L_\pm |l,m\rangle = \hbar \sqrt{l(l+1) - m(m\pm 1)} |l, m\pm 1\rangle \\
L_x = \frac{1}{2}(L_+ + L_-), \quad L_y = \frac{1}{2i}(L_+ - L_-)$.

\textbf{Spherical Harmonics $Y_l^m(\theta,\phi)$:}

\textbf{Coordinate Reps:}
$L_x = -i\hbar(y\partial_z - z\partial_y) = i\hbar(\sin\phi\partial_\theta + \cot\theta\cos\phi\partial_\phi) \\
L_y = -i\hbar(z\partial_x - x\partial_z) = i\hbar(-\cos\phi\partial_\theta + \cot\theta\sin\phi\partial_\phi) \\
L_z = -i\hbar(x\partial_y - y\partial_x) = -i\hbar\partial_\phi \\
L^2 = -\hbar^2 [\frac{1}{\sin\theta}\partial_\theta(\sin\theta\partial_\theta) + \frac{1}{\sin^2\theta}\partial_\phi^2] \\
Y_l^m(\theta,\phi) \propto P_l^m(\cos\theta) e^{im\phi}$. Parity: $Y_{l}^m(\hat{n}) = (-1)^l Y_l^m(-\hat{n})$

Conjugate: $Y_l^m{}^* = (-1)^m Y_l^{-m}$. 

Orthogonality: $\int d\Omega Y_{l'}^{m'*} Y_l^m = \delta_{ll'} \delta_{mm'}$.

Explicit: $Y_0^0 = \sqrt{\frac{1}{4\pi}}, Y_1^0 = \sqrt{\frac{3}{4\pi}}\cos\theta, Y_1^{\pm 1} = \mp\sqrt{\frac{3}{8\pi}}e^{\pm i\phi}\sin\theta$.

\subsection*{Addition of Angular Momentum}
$J=\frac{\hbar}{2mi}(\psi^\star\vec{\nabla}\psi-\psi\vec{\nabla}\psi^\star)=\frac{\hbar}{m}\Im{\psi^\star\vec{\nabla}\psi}$

connectivity: $\frac{d}{dt}|\psi|^2=-\vec{\nabla}\cdot\vec{J}$

$J = J_1 + J_2$. Basis $|j_1 j_2; m_1 m_2\rangle \to |j_1 j_2; j m\rangle$.
\important{|j_1 j_2; j m\rangle = \sum_{m_1, m_2} C_{m_1 m_2}^{j m} |j_1 j_2; m_1 m_2\rangle}
Selection: $m = m_1 + m_2, |j_1 - j_2| \le j \le j_1 + j_2$.

Example: Limit state $|j_1+j_2, j_1+j_2\rangle = |j_1, j_1\rangle |j_2, j_2\rangle$. lower with $J_-$:
$\alpha = \sqrt{\frac{j_1}{j_1+j_2}}, \quad \beta = -\sqrt{\frac{j_2}{j_1+j_2}}$.

\subsection*{Tensor Operators \& Wigner-Eckart}
Definition: $[J_z, T_k^q] = \hbar q T_k^q$

$[J_\pm, T_k^q] = \hbar \sqrt{k(k+1)-q(q\pm 1)} T_k^{q\pm 1}$.

\textbf{Wigner-Eckart Thm:} Matrix elements depend on $m$ only via CG coeff.
\important{\langle n' j' m' | T_k^q | n j m \rangle = \langle j k; m q | j' m' \rangle \frac{\langle n' j' || T_k || n j \rangle}{\sqrt{2j'+1}}}
Selection rules: $\Delta m = q$, Triangle rule for $j, k, j'$.
Example: $V \propto (x^2-y^2)$ is Tensor $T_2^{\pm 2} \implies \Delta m = \pm 2$.
Quadrupole: $Q_{ij} = \frac{V_i W_j + V_j W_i}{2} - \frac{\vec{V}\cdot\vec{W}}{3}\delta_{ij}$ (Traceless Symm Tensor).

\subsection*{Heisenberg Picture \& Dynamics}
\textbf{Schrödinger Eq:} $i\hbar \frac{d}{dt}|\psi\rangle = \mathcal{H}|\psi\rangle$.
\textbf{Time Independent:} $\mathcal{H}|\psi\rangle = E|\psi\rangle$.
\textbf{Heisenberg} $\frac{d}{dt} O_H = \frac{1}{i\hbar} [O_H, \mathcal{H}] + (\frac{\partial O}{\partial t})_H$.
\important{\frac{d}{dt}\langle O \rangle = \frac{1}{i\hbar}\langle [O, \mathcal{H}] \rangle + \langle \frac{\partial O}{\partial t} \rangle}

\textbf{Probability Current:} $\vec{J} = \frac{\hbar}{m}\text{Im}(\psi^*\nabla\psi) - \frac{q}{m}\vec{A}|\psi|^2$.

Definition: $O_H(t) = e^{i\mathcal{H}t/\hbar} O_S e^{-i\mathcal{H}t/\hbar}$.

Operator Calc: $\frac{d}{dt}e^{At} = A e^{At}, \quad \frac{d}{dt}(AB) = \dot{A}B + A\dot{B}$.

Example: Free particle $\dot{x} = p/m, \dot{p} = 0 \implies x(t) = x(0) + p(0)t/m$.

HO Solution: $x(t) = x(0)\cos(\omega t) + \frac{p(0)}{m\omega}\sin(\omega t)$.
$p(t) = p(0)\cos(\omega t) - m\omega x(0)\sin(\omega t)$.

\subsection*{Spherical Potentials}
Radial Eq: $u'' + [k^2 - \frac{l(l+1)}{r^2} - U(r)]u = 0$.
\textbf{Radial Momentum:} $p_r = -i\hbar(\partial_r + 1/r)$ (Hermitian).
Free particle ($U=0$): Spherical Bessel $j_l(kr)$ (regular at 0), Neumann $n_l(kr)$ (irregular).
Infinite Well ($0 < r < R$): $u(R)=0 \implies j_l(k_n R) = 0$.
$l=0$ states: $j_0(x) = \sin(x)/x \implies \sin(kR)=0 \implies k_n = n\pi/R$.

\subsection*{Atom in Magnetic Field}
\textbf{Gauge Invariance:} $\vec{A} \to \vec{A} + \nabla\Lambda, \phi \to \phi - \partial_t\Lambda$.
$\psi' = e^{iq\Lambda/\hbar}\psi$. Observables unchanged.
\textbf{Aharonov-Bohm:} Phase shift $\Delta\varphi = \frac{q}{\hbar} \oint \vec{A}\cdot d\vec{l} = \frac{q\Phi}{\hbar}$.
Interference shifts even if $\vec{B}=0$ on path (only $\vec{A} \neq 0$).
\textbf{Landau Level Degeneracy:} $N = \Phi / \Phi_0 = BA / (h/e)$.

$\mathcal{H} = \frac{1}{2m}(\mathbf{p} + e\mathbf{A})^2 - e\phi$. 
Gauge \textbf{Coulomb:} $\nabla \cdot \mathbf{A} = 0$. $\mathbf{A} = \frac{1}{2}\mathbf{B} \times \mathbf{r}$.
$\mathcal{H} \approx \mathcal{H}_0 + \mathcal{H}_P + \mathcal{H}_D$.

\textbf{Paramagnetic (Zeeman):} $\mathcal{H}_P = -\frac{e}{2m} \mathbf{L} \cdot \mathbf{B} = \mu_B \mathbf{L} \cdot \mathbf{B} / \hbar$.
\textbf{Diamagnetic:} $\mathcal{H}_D = \frac{e^2}{8m} (\mathbf{r} \times \mathbf{B})^2$.
\textbf{Gauge Trans:} $\vec{A}' = \vec{A} + \nabla\chi, \quad \phi' = \phi - \partial_t\chi, \quad \psi' = e^{iq\chi/\hbar}\psi$.





\subsection*{Perturbation Theory}
\textbf{Time-Independent Non-Degenerate} 

$\mathcal{H} = \mathcal{H}_0 + \lambda V$.
$E_n^{(1)} = V_{nn}$.

$|n^{(1)}\rangle = \sum_{k \neq n} \frac{V_{kn}}{E_n^{(0)} - E_k^{(0)}} |k^{(0)}\rangle$

$E_n^{(2)} = \sum_{k \neq n} \frac{|V_{kn}|^2}{E_n^{(0)} - E_k^{(0)}}$

\textbf{Time-Dependent:} $\mathcal{H}(t) = \mathcal{H}_0 + V(t)$. $c_n(t) = \langle n | \psi(t) \rangle e^{iE_n t/\hbar}$.
\important{\dot{c}_n(t) = -\frac{i}{\hbar} \sum_k V_{nk}(t) e^{i\omega_{nk}t} c_k(t)}
First Order: 

$c_n^{(1)}(t) = -\frac{i}{\hbar} \int_{t_0}^t \langle n | V(t') | i \rangle e^{i\omega_{ni}t'} dt'$.
 Fermi's Golden Rule (Transition Rate $i \to f$):
\important{W_{i\to f} = \frac{2\pi}{\hbar} |V_{fi}|^2 \delta(E_f - E_i)}

\subsection*{Identical Particles}
\textbf{Exchange Symmetry:} $P_{12}\psi(1,2) = \eta\psi(2,1) = \eta\psi(1,2)$.
$\eta=1$ (Bosons, Int Spin), $\eta=-1$ (Fermions, Half-Int Spin).
\textbf{Pauli Exclusion:} 2 Fermions cannot occupy same state.
\textbf{Symmetrizers:} $S_N = \frac{1}{N!} \sum P_g, \quad A_N = \frac{1}{N!} \sum (-1)^p P_g$.
Helium Atom: 

$\mathcal{H} = \frac{p_1^2}{2m} + \frac{p_2^2}{2m} - \frac{2e^2}{r_1} - \frac{2e^2}{r_2} + \frac{e^2}{r_{12}}$.

Treat $e^2/r_{12}$ as perturbation.
Ground State ($1s^2$): Singlet ($S=0$, anti-sym spin) $\implies$ Sym spatial.
Excited ($1s2s$):

- Parahelium (Singlet): $E_S = E_{1s} + E_{2s} + J + K$.

- Orthohelium (Triplet): $E_T = E_{1s} + E_{2s} + J - K$.
Exchange energy $K > 0$, so Triplet is lower (Hund's Rule \#1).

\subsection*{Scattering - Born Approx}
Lippmann-Schwinger: 

$|\psi\rangle = |\phi\rangle + G_0^+ V |\psi\rangle$.

Born Series: 

$|\psi\rangle = |\phi\rangle + G_0^+ V |\phi\rangle + \dots$

\textbf{First Born Approximation:}
$f(\theta, \phi) = -\frac{m}{2\pi\hbar^2} \int d^3r e^{-i\Delta\mathbf{k}\cdot\mathbf{r}} V(\mathbf{r}) = -\frac{m}{2\pi\hbar^2} \tilde{V}(\mathbf{q})$
where $\mathbf{q} = \mathbf{k}' - \mathbf{k}$ is momentum transfer. $|q| = 2k\sin(\theta/2)$.
\textbf{Form Factor:} $f(q) = F(q^2) f_{point}(q)$. $F(q^2) = \int d^3r \rho(r) e^{i\vec{q}\cdot\vec{r}}$.
Radius: $R^2 = -6 \frac{dF}{dq^2}|_{q=0}$.
\textbf{Delta Perturbation:} $V(\vec{r}) = \alpha\delta(\vec{r}) \implies \Delta E = \alpha |\psi(0)|^2$ (Only $s$-waves).

\subsection*{Jacobi Transformation}
S first to one before last row is distance between particles, last row in distance between particle of center of mass.
example: first row is distance between the first pair, second row is distance between the third and the two first and so forth, last is to center of mass

$\eta_A=R=\frac{1}{\sqrt{A}}\sum_{i=1}^A r_i$

\important{\eta_k=\sqrt{\frac{k}{k+1}}(r_{k+1}-\frac{1}{k}\sum^k_{i=1}r_i)}

values of row k, set $r_i$ to be distances to the relavant particles (first pair use only $r_1,r_2$, third and first $r_3,r_1$ rest are zero)


\section*{MATH}

\subsection*{Operators Definitions}

$|\psi\rangle\langle\psi|=I, A = \sum_{ij}|u_i\rangle\langle u_j| \\
C_{ij} = \sum_{k=1}^{n} A_{ik} B_{kj}, \langle A\rangle_{\alpha\beta}=\langle A\rangle^\dagger_{\beta\alpha}$

\textbf{Linear:} $A(c_1\psi + c_2\phi) = c_1 A\psi + c_2 A\phi$.
\textbf{Anti-linear:} $A(c_1\psi + c_2\phi) = c_1^* A\psi + c_2^* A\phi$.

Preserves probability: $|\langle A\phi | A\psi \rangle| = |\langle \phi | \psi \rangle|$.
Identity: $\langle A\phi | A\psi \rangle = \langle \phi | \psi \rangle^*$.

\textbf{Hermitian:} $X = X^\dagger$. Real eigenvalues, orthogonal eigenstates.
Property: $\langle \alpha | X | \beta \rangle^* = \langle \beta | X^\dagger | \alpha \rangle$.
If Hermitian: $\langle \alpha | X | \beta \rangle^* = \langle \beta | X | \alpha \rangle$.

\textbf{Unitary:} $U^\dagger U = U U^\dagger = I$. Preserves inner product.
\important{\langle U\psi | U\phi \rangle = \langle \psi | \phi \rangle}
Form: $U = e^{iX}$ where $X$ is Hermitian.

self values are $\lambda=e^{i\theta}$

\textbf{Anti-Unitary:} Operator $A$ is anti-unitary if:
1.

\textbf{Hermitic:} $\langle Aa|b\rangle=\langle a|Ab\rangle$

$\langle Aa|b\rangle = \int (Aa)^\star b (yakobian) dr$

$\langle a|Ab\rangle = \int a^\star A[b (yakobian)] dr$

\subsection*{Levi-Civita Symbol ($\epsilon_{ijk}$)}
$\epsilon_{ijk} = \begin{cases} +1 \quad 123,231,312\\ -1 \quad 321,132,213 \\ 0 \quad else \end{cases}$

$(\mathbf{A} \times \mathbf{B})_i = \epsilon_{ijk} A_j B_k$

$\sum_{i=1}^3\epsilon_{ijk} \epsilon_{ilm} = \delta_{jl}\delta_{km} - \delta_{jm}\delta_{kl}$

\subsection*{Commutator Identities}
$[A, BC] = [A, B]C + B[A, C]$

$[AB, C] = A[B, C] + [A, C]B$

$[A, [B, C]] + [B, [C, A]] + [C, [A, B]] = 0$

$[A,A] = 0, [A, e^{cA}] = 0$ 

$[f(A), A] = 0$ 

$[[A,B],B]=0\rightarrow[A, f(B)] = [A, B]f'(B)$

$[P, F(X)] = -i\hbar \frac{\partial F}{\partial X}, [X, G(P)] = i\hbar \frac{\partial G}{\partial P}$

$[A,B] = 0 \rightarrow  e^A e^B = e^{A+B+\frac{1}{2}[A,B] + \dots}$

$e^A B e^{-A} = B + [A,B] + \frac{1}{2!}[A,[A,B]] + \dots$

$[L_i,L_j] = i\hbar\epsilon_{idk}L_k \quad [\mathcal{H}, L] = 0 $

\textbf{Uncertainty:} $\Delta A \Delta B \ge \frac{1}{2}|\langle [A,B] \rangle|$.

\textbf{Canonical:} $[x_i, p_j] = i\hbar \delta_{ij}$.
From Ex: $[L_i, L_j] = i\hbar \epsilon_{ijk} L_k$.

$[L_i, x_j] = i\hbar \epsilon_{ijk} x_k, \quad [L_i, p_j] = i\hbar \epsilon_{ijk} p_k$.

$[L_i x_k, L_j p_m]$ via standard expansion.

$[a^\dagger a, a] = -a, [a^\dagger a, a^\dagger] = a^\dagger$.

$[X_{ab}, X_{cd}] = i \hbar[ \delta_{bc} X_{ad} - \delta_{bd} X_{ac} - \delta_{ac} X_{bd} + \delta_{ad} X_{bc}]$

$[J_i, J_j] = i\hbar \epsilon_{ijk} J_k, \quad [L^2, L_i] = 0$
$[L_i, x_j] = i\hbar \epsilon_{ijk} x_k, \quad [L_i, p_j] = i\hbar \epsilon_{ijk} p_k$

$[J^2,J_c]=0 \quad  [J_+,J_-]=2J_0$

$[J_0,J_\pm]=\pm J_\pm$

$[J_i, J_j] = i\hbar \epsilon_{ijk} J_k, \quad [L^2, L_i] = 0$.
$[L_i, x_j] = i\hbar \epsilon_{ijk} x_k, \quad [L_i, p_j] = i\hbar \epsilon_{ijk} p_k$.


\textbf{specific shit}

$M_i = X_{i4}$

$[L_i, M_j]=i\hbar\epsilon_{ijk}M_k, [M_i,M_j] = i\hbar\epsilon_{ijk}L_k$

\underbar{$G_i^\pm=\frac{1}{2}(L_i\pm M_i)$}

$[G_i^\pm,G_j^\pm]=i\hbar\epsilon_{ijk}G_k^\pm $

$[G_k^+,G_j^-]=0$

\underbar{LaplaceRungeLentz $A=\frac{1}{m}p\times L-k\hat{r}$}

$[L_i,A_j]=i\hbar\epsilon_{ijk}A_k$

$[A_i,A_j]=-2im\mathcal{H}\hbar\epsilon_{ijk}L_k$

$K=\frac{A}{\sqrt(-2mH}\quad$ K is preserved

$[K_i,K_j]=-i\hbar\epsilon_{ijk}L_k$

\underbar{$A=-\frac{1}{2}r\times B$}

$\Pi_i = m\dot{r} = p_i + \frac{q}{2}\epsilon_{ijk}r_j B_k$

$[\prod_i,\prod_j] = i\hbar\epsilon_{ijk}B_k$

$\dot{\prod_i} = -\frac{i}{\hbar}[\prod_i,\mathcal{H}] = qv\times V$ Lorentz


\subsection*{Symmetry Generators}
Time: $T(t) = e^{-iHt/\hbar}$ (Energy).

Space: $U(\vec{a}) = e^{-i\vec{a}\cdot\vec{P}/\hbar}$ (Momentum).

Rotation: $R(\hat{n}, \theta) = e^{-i\theta \hat{n}\cdot\vec{J}/\hbar}$ (Ang. Mom).

$|x+\alpha\rangle=U(x)|\alpha\rangle$

\subsection*{Pauli Matrices ($\sigma$)}
$\sigma_x = \begin{pmatrix} 0 & 1 \\ 1 & 0 \end{pmatrix}, \sigma_y = \begin{pmatrix} 0 & -i \\ i & 0 \end{pmatrix}, \sigma_z = \begin{pmatrix} 1 & 0 \\ 0 & -1 \end{pmatrix}$.
\important{\sigma_i \sigma_j = \delta_{ij} I + i\epsilon_{ijk} \sigma_k}
$[\sigma_i, \sigma_j] = 2i\epsilon_{ijk} \sigma_k \quad \{\sigma_i, \sigma_j\} = 2\delta_{ij} I$

Identity: $(\vec{\sigma}\cdot\vec{a})(\vec{\sigma}\cdot\vec{b}) = (\vec{a}\cdot\vec{b})I + i\vec{\sigma}\cdot(\vec{a}\times\vec{b})$

Rotation: $e^{-i\frac{\theta}{2}\hat{n}\cdot\vec{\sigma}} = I\cos\frac{\theta}{2} - i(\hat{n}\cdot\vec{\sigma})\sin\frac{\theta}{2}$.

$\sigma_i\sigma_j+\sigma_j\sigma_i=2\delta_{ij}I $

\subsection*{Position \& Momentum}
Fourier: $\psi(x) = \frac{1}{\sqrt{2\pi\hbar}} \int dp \phi(p) e^{ipx/\hbar}$.
\textbf{Harmonic Oscillator Ladder Ops:}
$N = a^\dagger a$

$a = \sqrt{\frac{m\omega}{2\hbar}} (x + \frac{i}{m\omega}p)$

$a^\dagger = \sqrt{\frac{m\omega}{2\hbar}} (x - \frac{i}{m\omega}p)$

$x = \sqrt{\frac{\hbar}{2m\omega}}(a + a^\dagger) = -i\hbar\vec{\nabla}_p$

$p = i\sqrt{\frac{m\hbar\omega}{2}}(a^\dagger - a) = -i\hbar\vec{\nabla}$

$\frac{\vec{p}^2}{2m}=-\frac{\hbar^2}{2m}\vec{\nabla}^2$

$\mathcal{H} = \hbar\omega(a^\dagger a + 1/2) = \hbar\omega(N + 1/2)$

$[a, a^\dagger] = 1, [N, a] = -a, [N, a^\dagger] = a^\dagger$

\textbf{Virial Theorem:} $2\langle T \rangle = n\langle V \rangle$ for $V(x) \propto x^n$.
HO $(n=2) \implies \langle T \rangle = \langle V \rangle = E/2$.

\subsection*{Math Supplement}
$\cos a \cos b = \frac{1}{2}[\cos(a-b)+\cos(a+b)]$
$\sin a \sin b = \frac{1}{2}[\cos(a-b)-\cos(a+b)]$
$\sin a \cos b = \frac{1}{2}[\sin(a-b)+\sin(a+b)]$
$\sin(a\pm b) = \sin a\cos b \pm \cos a\sin b$
$\cos(a\pm b) = \cos a\cos b \mp \sin a\sin b$
$\cos^2 x = \frac{1+\cos 2x}{2}, \sin^2 x = \frac{1-\cos 2x}{2}$

\textbf{Taylor series:}

$e^x = \sum \frac{x^n}{n!} \approx 1 + x + \frac{1}{2} x^2 \\
\sin x = \sum \frac{(-1)^n x^{2n+1}}{(2n+1)!} \approx x - \frac{1}{3!} x^3\\
\cos x = \sum \frac{(-1)^n x^{2n}}{(2n)!} \approx 1 - \frac{1}{2}x^2 \\
\ln(1+x) = x - \frac{1}{2}x^2 + \frac{1}{3}x^3 \\
(1+x)^\alpha = 1 +\alpha x + \frac{\alpha(\alpha-1)}{2}x^2$

\textbf{Delta:} $\delta(f(x)) = \sum_i \frac{\delta(x-x_i)}{|f'(x_i)|}$

$\delta^{(3)}(\vec{r}-\vec{r}_0) = \frac{1}{r^2\sin\theta}\delta(r-r_0)\delta(\theta-\theta_0)\delta(\phi-\phi_0)$.

\underbar{\textbf{Quantization:}}

\important{Poisson: \quad \{f,g\} \to \frac{1}{i\hbar}[F,G]}
\important{Replaces: \quad xp \to \frac{1}{2}[X,P]}

when quantifying something, try to make the math fit one of the two formulas, believe in yourself.

remember: $(A\times B)^\dagger=-B\times A$

example:

$A = \frac{1}{m} p \times L - k\hat{r} \\$
$p \times L = \frac{1}{2}(p \times L + p \times L) \\$
$p \times L = \frac{1}{2}(p \times L - L \times p) \\$
$p \times L = \frac{1}{2}\{p, L\} \\$
$p \times L = \frac{1}{2} \frac{1}{i\hbar} [P, L] \\$
$A = \frac{1}{2im\hbar} [P, L] - k\hat{r}$

\textbf{Expectation:} 

$\langle f(X) \rangle = \int_{-\infty}^\infty f(x) |\psi(x)|^2 dx$.

$\langle \alpha |f|\beta \rangle = \int_{0}^{\infty} \psi_{\alpha}^*(r) \left[ \hat{f} \psi_{\beta}(r) \right] r^2 dr = \sum_{i,j} a_i^* F_{ij} b_j$


$\langle x'|x\rangle=\delta_{xx'}$

\textbf{Volume Elements:} $dV_{cyl} = \rho d\rho d\theta dz$. $dV_{sph} = r^2 \sin\theta dr d\theta d\phi$.

\textbf{Integrals:}
$\int_{-\infty}^\infty e^{-ax^2+bx} dx = \sqrt{\frac{\pi}{a}}e^{b^2/4a} \\
\int \sin^2(ax) dx = \frac{x}{2} - \frac{\sin(2ax)}{4a} \\ 
\int \cos^2(ax) dx = \frac{x}{2} + \frac{\sin(2ax)}{4a} \\ 
\int_0^\infty x^n e^{-ax} dx = n!/a^{n+1}$.


\end{multicols*}

\end{document}
